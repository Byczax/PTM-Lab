%!TEX program = xelatex
%Template created by: Maciej Byczko
\documentclass[a4paper,12pt]{extarticle}  %typ dokumentu

% \usepackage[utf8]{inputenc} %rodzaj czcionki w dokumencie
\usepackage{geometry} %poprawienie marginesów
\usepackage{polski} %polskie znaki
\usepackage{multirow} %tabela
\usepackage{graphicx} %tabela
\usepackage{float} %poprawienie pozycji
\usepackage{amsmath} % Matma
\usepackage{fancyhdr} % header i footer
\usepackage{xcolor} %kolorowy tekst
\usepackage{boldline}%edytowanie grubości krawędzi w tabelach \hlineB{} \clineB{}{}
\usepackage{listings} %pisanie kodu w ładny sposób, begin{listings}[language=<język>]...end{listings} tak samo jak nazwa paczki

\usepackage{hyperref} %tworzenie odnośników, \url{<url>}, \href{<file path, link>}{<text with link>} \pageref{}

%Ustawienie paczki hyperref
\hypersetup{
     colorlinks,
     citecolor=black,
     filecolor=black,
     linkcolor=black,
     urlcolor=black
}

\definecolor{backcolour}{rgb}{0.95,0.95,0.92}
\definecolor{AO}{rgb}{0,0.5,0}
\definecolor{ZeroBlue}{rgb}{0,0.28,0.73}
\definecolor{DarkRed}{rgb}{0.85,0.16,0.16}

\lstdefinelanguage
   [x64]{Assembler}     % add a "x64" dialect of Assembler
   [x86masm]{Assembler} % based on the "x86masm" dialect
   % nie wolno zamieniać kolejności deklaracji morekeywords
   {
    deletekeywords={mov,inc,org, end, nop, jmp, jnb, jnc, pop, jz, push, ret, jb, jnz, setb, add, mul},
    morekeywords={[3]a,b,r0,r1,r2,r3,r4,r5,r6,r7,\$,c,dptr, @dptr, @a, 00, acc, ab, p1, p2, p3 ,p4, p5, p6 ,p7 , p8, p9, dpl, dph},
    morekeywords={[2]subb, clr, ljmp, movx,end,mov,inc, nop, jmp, end, orl, jnb, jnc, djnz, sjmp, pop, acall, jz, push, ret, jb, jnz, anl, cpl, setb, add, mul, rrc, rr}   
    }

                
\lstset{
     breaklines=true,
     language=[x64]Assembler,
     numbers=left,
     numberstyle=\tiny,
     tabsize=2,
     morecomment=[s]{/*}{*/},
     morecomment=[l]{//},
     backgroundcolor=\color{backcolour},
     breakatwhitespace=false,
     showspaces=false,                
     showstringspaces=false,
     showtabs=false, 
     commentstyle=\color{gray},
     keywordstyle=\color{AO},
     keywordstyle={[2]\color{DarkRed}},
     keywordstyle={[3]\color{ZeroBlue}},
}

\graphicspath{{pictures/}}
\geometry{margin=0.6in}
\pagestyle{fancy}
\cfoot{Strona \thepage}
\rhead{Strona \thepage}
\lhead{\typdoc}
\setlength{\headheight}{15pt}
\newcolumntype{?}{!{\vrule width 1.5pt}}

\title{\tytul}
\author{\tworcy}
\date{\data}

\newcommand*\circled[1]{\tikz[baseline=(char.base)]{
\node[shape=circle,draw,inner sep=1pt] (char) {#1};}}

%-----------------------SEKCJA DANYCH----------------------------------
\def\tytul{Transmisja portem szeregowym} %<<< tytuł ćwiczenia
\def\nrcw{6} %<<< numer ćwiczenia
\def\data{6 Maja 2021} %<< data wykonania
\def\prowadzacy{dr inż. Jacek Mazurkiewicz} %<<<prowadzący
\def\nrgrupy{B} %<<<numer grupy
\def\tworcy{Maciej Byczko\\Bartosz Matysiak} %<<< autorzy
\def\zajinfo{Cz 13:15 TN} %<<< informacje dotyczące zajęć
\def\typdoc{Sprawozdanie - \tytul} %<<< typ dokumentu tj Sprawozdanie, zadania itp. {Matematyka dyskretna/Sprawozdanie z Miernictwa}
% \tableofcontents % Stworzenie spis treści
%JEŻELI COS JESZCZE POTRZEBA W TEJ SEKCJI TO POINFORMOWAĆ!!!
%----------------------------------------------------------------------

%-----------------------SEKCJA FORMATOWANIA----------------------------
% \textbf{pogrubienie}  \textit{kursywa}    \underline{podkreślenie}
% \vspace*{2mm} - Odstęp pionowy między tekstem
%----------------------------------------------------------------------

\begin{document}
%-------------------------------------TABELA-DANYCH--------------------------------------------------
\begin{table}[H]
	\centering
	\resizebox{\textwidth}{!}{
		\begin{tabular}{|c|c|c|}\hline
			\begin{tabular}[c]{@{}c@{}}                     \tworcy     \end{tabular} &
			\begin{tabular}[c]{@{}c@{}}Prowadzący:\\        \prowadzacy \end{tabular} &
			\begin{tabular}[c]{@{}c@{}}Numer ćwiczenia\\    \nrcw       \end{tabular}          \\ \hline
			\begin{tabular}[c]{@{}c@{}}                     \zajinfo    \end{tabular} &
			\begin{tabular}[c]{@{}c@{}}Temat ćwiczenia:\\   \tytul      \end{tabular} & Ocena: \\ \hline
			\begin{tabular}[c]{@{}c@{}}Grupa:   \\          \nrgrupy    \end{tabular} &
			\begin{tabular}[c]{@{}c@{}}Data wykonania:\\    \data       \end{tabular} &        \\ \hline
		\end{tabular}}
\end{table}
%----------------------------------------------------------------------------------------------------
\section{Polecenie}
We wszystkich programach prezentowanych na zajęciach występowało "zjawisko" przekształcania danych
przychodzących z terminala na komputerze tak by reprezentowały rzeczywiście kody ASCII znaków.
W ten sposób udało się rozwiązać problem, a w zasadzie zaadaptować się do nieznanej przyczyny, stałego
"zniekształcenia - przekłamania" danych odbieranych portem szeregowym. Zadanie polega na analizie
tego fragmentu kodu, w którym owe przekształcenie danych jest realizowane i opisaniu jakie działania są
w nim podejmowane, na czym polega to "przekłamanie" danych i - ewentualnie - zaproponowanie innego
- może lepszego / sprawniejszego - sposobu zrealizowania tej koniecznej korekty.
\section{Analiza oryginalnego kodu}
\lstinputlisting{kod.asm}
W kolejnych instrukcjach zawartości rejestrów zmieniają się następująco:\\
\emph{(symbole $A_7-A_0$ oznaczają wartości bitów stanowiących początkową zawartość akumulatora, od najstarszego do najmłodszego)}

\begin{table}[H]
	\centering
	\resizebox{\textwidth}{!}{
		\begin{tabular}{ccccccccccc}
			$(1)$ & $R_7|A$ & $=$ & $A_7$ & $A_6$ & $A_5$ & $A_4$ & $A_3$ & $A_2$ & $A_1$ & $A_0$ \\
			$(2)$ & $R_7$ & $=$ & $  0$ & $  0$ & $  0$ & $  0$ & $A_3$ & $A_2$ & $A_1$ & $A_0$ \\
			$(3)$ & $A  $ & $=$ & $  0$ & $A_7$ & $A_6$ & $A_5$ & $A_4$ & $A_3$ & $A_2$ & $A_1$ \\
			$(4)$ & $A  $ & $=$ & $  0$ & $A_7$ & $A_6$ & $A_5$ & $  0$ & $  0$ & $  0$ & $  0$ \\
			$(5)$ & $R_7$ & $=$ & $  0$ & $A_7$ & $A_6$ & $A_5$ & $A_3$ & $A_2$ & $A_1$ & $A_0$
		\end{tabular}
	}
\end{table}
W wyniku podanej konwersji, młodszy półbajt akumulatora pozostał bez zmian, natomiast w starszym półbajcie trzy najstarsze bity uległy przesunięciu na młodszą pozycję.
Na pozycji najstarszej pojawiło się \textbf{0}, a wartość \textbf{$A_4$} uległa destrukcji.
\clearpage
\section{Alternatywny kod}
Kod można uprościć o jedną instrukcję i pozbyć się tym samym wykorzystania bitu carry, zamieniając kolejność instrukcji:
\lstinputlisting{alter.asm}
\begin{table}[H]
	\centering
	\resizebox{\textwidth}{!}{
		\begin{tabular}{ccccccccccc}
			$(1)$ & $R_7|A$ & $=$ & $A_7$ & $A_6$ & $A_5$ & $A_4$ & $A_3$ & $A_2$ & $A_1$ & $A_0$ \\
			$(2)$ & $R_7$ & $=$ & $  0$ & $  0$ & $  0$ & $  0$ & $A_3$ & $A_2$ & $A_1$ & $A_0$ \\
			$(3)$ & $A  $ & $=$ & $A_7$ & $A_6$ & $A_5$ & $  0$ & $  0$ & $  0$ & $  0$ & $  0$ \\
			$(4)$ & $A  $ & $=$ & $  0$ & $A_7$ & $A_6$ & $A_5$ & $  0$ & $  0$ & $  0$ & $  0$ \\
			$(5)$ & $R_7$ & $=$ & $  0$ & $A_7$ & $A_6$ & $A_5$ & $A_3$ & $A_2$ & $A_1$ & $A_0$
		\end{tabular}
	}
\end{table}
\section{Alternatywny kod 2}
Kod ASCII wykorzystuje siedem młodszych bitów bajtu.
Wszystkie bajty wykorzystane do reprezentacji mają na najstarszej pozycji bit o wartości 0.
Kody powyżej realizują wyzerowanie najstarszego bitu, a także przesunięcie bitowe starszej części.\\
O ile przesunięcie bitowe można zaniedbać, najstarszy bit akumulatora można wyzerować jedną instrukcją:
\begin{lstlisting}
                              CLR ACC.7
\end{lstlisting}
% Kod wykonuje następujące operacje na akumulatorze:
% \begin{itemize}
%      \item[2] Pozostaw 4 młodsze bity bez zmian
%      \item[$3, 4$] trzy najstarsze bity przesuń na miejsce swojego młodszego "sąsiada"
%      \item[5] Operacja "niszczy" wartość piątego bitu ($A_4$) a na najstarszej pozycji umieszcza zero.
% \end{itemize}


\end{document}
