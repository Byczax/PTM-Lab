%!TEX program = xelatex
%Template created by: Maciej Byczko
\documentclass[a4paper,12pt]{extarticle}  %typ dokumentu

\usepackage[utf8]{inputenc} %rodzaj czcionki w dokumencie
\usepackage{geometry} %poprawienie marginesów
\usepackage{polski} %polskie znaki
\usepackage{multirow} %tabela
\usepackage{graphicx} %tabela
\usepackage{float} %poprawienie pozycji
\usepackage{amsmath} % Matma
\usepackage{fancyhdr} % header i footer
\usepackage{xcolor} %kolorowy tekst
\usepackage{boldline}%edytowanie grubości krawędzi w tabelach \hlineB{} \clineB{}{}
\usepackage{listings} %pisanie kodu w ładny sposób, begin{listings}[language=<jezyk>]...end{listings} tak samo jak nazwa paczki

\usepackage{hyperref} %tworzenie odnośników, \url{<url>}, \href{<file path, link>}{<text with link>} \pageref{}

%Ustawienie paczki hyperref
\hypersetup{
     colorlinks,
     citecolor=black,
     filecolor=black,
     linkcolor=black,
     urlcolor=black
}

\definecolor{backcolour}{rgb}{0.95,0.95,0.92}

\lstset{
breaklines=true,
language=[x86masm]Assembler,
numbers=left,
numberstyle=\tiny,
tabsize=2,
backgroundcolor=\color{backcolour},
breakatwhitespace=false,
showspaces=false,                
showstringspaces=false,
showtabs=false, 
}

\graphicspath{{pictures/}}
\geometry{margin=0.7in}
\pagestyle{fancy}
\cfoot{Strona \thepage}
\rhead{Strona \thepage}
\lhead{\typdoc}
\newcolumntype{?}{!{\vrule width 1.5pt}}

\title{\tytul}
\author{\tworcy}
\date{\data}

\newcommand*\circled[1]{\tikz[baseline=(char.base)]{
\node[shape=circle,draw,inner sep=1pt] (char) {#1};}}

%-----------------------SEKCJA DANYCH----------------------------------
\def\tytul{Arytmetyka, logika, pamięć, diody i brzęczyki} %<<< tytuł ćwiczenia
\def\nrcw{2} %<<< numer ćwiczenia
\def\data{11 Marca 2021} %<< data wykonania
\def\prowadzacy{dr inż. Jacek Mazurkiewicz} %<<<prowadzący
\def\nrgrupy{B} %<<<numer grupy
\def\tworcy{Maciej Byczko\\Bartosz Matysiak} %<<< autorzy
\def\zajinfo{Cz 13:15 TN} %<<< informacje dotyczące zajęć
\def\typdoc{Sprawozdanie} %<<< typ dokumentu tj Sprawozdanie, zadania itp. {Matematyka dyskretna/Sprawozdanie z Miernictwa}
% \tableofcontents % Stworzenie spis treści
%JEŻELI COS JESZCZE POTRZEBA W TEJ SEKCJI TO POINFORMOWAĆ!!!
%----------------------------------------------------------------------

%-----------------------SEKCJA FORMATOWANIA----------------------------
% \textbf{pogrubienie}  \textit{kursywa}    \underline{podkreślenie}
% \vspace*{2mm} - Odstęp pionowy między tekstem
%----------------------------------------------------------------------

\begin{document}
%-------------------------------------TABELA-DANYCH--------------------------------------------------
    \begin{table}[H]
        \centering
        \resizebox{\textwidth}{!}{
        \begin{tabular}{|c|c|c|}\hline
            \begin{tabular}[c]{@{}c@{}}                     \tworcy     \end{tabular}&
            \begin{tabular}[c]{@{}c@{}}Prowadzący:\\        \prowadzacy \end{tabular}&
            \begin{tabular}[c]{@{}c@{}}Numer ćwiczenia\\    \nrcw       \end{tabular}        \\ \hline
            \begin{tabular}[c]{@{}c@{}}                     \zajinfo    \end{tabular}&
            \begin{tabular}[c]{@{}c@{}}Temat ćwiczenia:\\   \tytul      \end{tabular}& Ocena:\\ \hline
            \begin{tabular}[c]{@{}c@{}}Grupa:   \\          \nrgrupy    \end{tabular}& 
            \begin{tabular}[c]{@{}c@{}}Data wykonania:\\    \data       \end{tabular}&       \\ \hline
        \end{tabular}}
    \end{table}
%----------------------------------------------------------------------------------------------------
\section{Zadanie 1}
\subsection{Polecenie}
Przygotować program odejmujący dwie liczby 16-bitowe.
\subsection{Rozwiązanie}
\lstinputlisting{zad1.a51}

\section{Zadanie 2}
\subsection{Polecenie}
Przygotować program realizujący sortowanie bąbelkowe lub znajdujący minimum albo maksimum
tablicy 1-wymiarowej rozpoczynającej się od adresu 8000H pamięci zewnętrznej danych (XRAM)
i obejmującej 16 kolejnych komórek tej tablicy.
\subsection{Rozwiązanie}

\section{Zadanie 3}
\subsection{Polecenie}
Przygotować program realizujący ciekawe zapalanie/gaszenie diód podłączonych do portu P1.
\subsection{Rozwiązanie}

\end{document}