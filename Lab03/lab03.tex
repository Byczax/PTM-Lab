%!TEX program = xelatex
%Template created by: Maciej Byczko
\documentclass[a4paper,12pt]{extarticle}  %typ dokumentu

% \usepackage[utf8]{inputenc} %rodzaj czcionki w dokumencie
\usepackage{geometry} %poprawienie marginesów
\usepackage{polski} %polskie znaki
\usepackage{multirow} %tabela
\usepackage{graphicx} %tabela
\usepackage{float} %poprawienie pozycji
\usepackage{amsmath} % Matma
\usepackage{fancyhdr} % header i footer
\usepackage{xcolor} %kolorowy tekst
\usepackage{boldline}%edytowanie grubości krawędzi w tabelach \hlineB{} \clineB{}{}
\usepackage{listings} %pisanie kodu w ładny sposób, begin{listings}[language=<język>]...end{listings} tak samo jak nazwa paczki

\usepackage{hyperref} %tworzenie odnośników, \url{<url>}, \href{<file path, link>}{<text with link>} \pageref{}

%Ustawienie paczki hyperref
\hypersetup{
     colorlinks,
     citecolor=black,
     filecolor=black,
     linkcolor=black,
     urlcolor=black
}

\definecolor{backcolour}{rgb}{0.95,0.95,0.92}
\definecolor{AO}{rgb}{0,0.5,0}
\definecolor{ZeroBlue}{rgb}{0,0.28,0.73}
\definecolor{DarkRed}{rgb}{0.85,0.16,0.16}

\lstdefinelanguage
   [x64]{Assembler}     % add a "x64" dialect of Assembler
   [x86masm]{Assembler} % based on the "x86masm" dialect
   % nie wolno zamieniać kolejności deklaracji morekeywords
   {
    deletekeywords={mov,inc,org, end, nop, jmp, jnb, jnc, pop, jz, push, ret, jb},
    morekeywords={[3]a,b,r0,r1,r2,r3,r4,r5,r6,r7,\$,c,dptr, @dptr, @a, 00, acc},
    morekeywords={[2]subb, clr, ljmp, movx,end,mov,inc, nop, jmp, end, orl, jnb, jnc, djnz, sjmp, pop, acall, jz, push, ret, jb}   
    }

                
\lstset{
     breaklines=true,
     language=[x64]Assembler,
     keywordsprefix=\#,
     alsoletter=\#, 
     numbers=left,
     numberstyle=\tiny,
     tabsize=2,
     backgroundcolor=\color{backcolour},
     breakatwhitespace=false,
     showspaces=false,                
     showstringspaces=false,
     showtabs=false, 
     commentstyle=\color{gray},
     keywordstyle=\color{AO},
     keywordstyle={[2]\color{DarkRed}},
     keywordstyle={[3]\color{ZeroBlue}},
}

\graphicspath{{pictures/}}
\geometry{margin=0.7in}
\pagestyle{fancy}
\cfoot{Strona \thepage}
\rhead{Strona \thepage}
\lhead{\typdoc}
\setlength{\headheight}{15pt}
\newcolumntype{?}{!{\vrule width 1.5pt}}

\title{\tytul}
\author{\tworcy}
\date{\data}

\newcommand*\circled[1]{\tikz[baseline=(char.base)]{
\node[shape=circle,draw,inner sep=1pt] (char) {#1};}}

%-----------------------SEKCJA DANYCH----------------------------------
\def\tytul{Klawiatura} %<<< tytuł ćwiczenia
\def\nrcw{4} %<<< numer ćwiczenia
\def\data{8 Kwietnia 2021} %<< data wykonania
\def\prowadzacy{dr inż. Jacek Mazurkiewicz} %<<<prowadzący
\def\nrgrupy{B} %<<<numer grupy
\def\tworcy{Maciej Byczko\\Bartosz Matysiak} %<<< autorzy
\def\zajinfo{Cz 13:15 TN} %<<< informacje dotyczące zajęć
\def\typdoc{Sprawozdanie} %<<< typ dokumentu tj Sprawozdanie, zadania itp. {Matematyka dyskretna/Sprawozdanie z Miernictwa}
% \tableofcontents % Stworzenie spis treści
%JEŻELI COS JESZCZE POTRZEBA W TEJ SEKCJI TO POINFORMOWAĆ!!!
%----------------------------------------------------------------------

%-----------------------SEKCJA FORMATOWANIA----------------------------
% \textbf{pogrubienie}  \textit{kursywa}    \underline{podkreślenie}
% \vspace*{2mm} - Odstęp pionowy między tekstem
%----------------------------------------------------------------------

\begin{document}
%-------------------------------------TABELA-DANYCH--------------------------------------------------
\begin{table}[H]
	\centering
	\resizebox{\textwidth}{!}{
		\begin{tabular}{|c|c|c|}\hline
			\begin{tabular}[c]{@{}c@{}}                     \tworcy     \end{tabular} &
			\begin{tabular}[c]{@{}c@{}}Prowadzący:\\        \prowadzacy \end{tabular} &
			\begin{tabular}[c]{@{}c@{}}Numer ćwiczenia\\    \nrcw       \end{tabular}          \\ \hline
			\begin{tabular}[c]{@{}c@{}}                     \zajinfo    \end{tabular} &
			\begin{tabular}[c]{@{}c@{}}Temat ćwiczenia:\\   \tytul      \end{tabular} & Ocena: \\ \hline
			\begin{tabular}[c]{@{}c@{}}Grupa:   \\          \nrgrupy    \end{tabular} &
			\begin{tabular}[c]{@{}c@{}}Data wykonania:\\    \data       \end{tabular} &        \\ \hline
		\end{tabular}}
\end{table}
%----------------------------------------------------------------------------------------------------
\section{Zadanie 1}
\subsection{Polecenie}
uzyskać 8-bitowy kod skaningowy klawisza:
kod uzyskuje się poprzez:
\begin{itemize}
    \item uaktywnienie odpowiedniej linii wiersza:
    \begin{itemize}
        \item stan 0 na linii portu P5:od P5.4 do P5.7
        \item odczytanie stanu młodszej części portu P7: linie od P7.0 do P7.3
        \item złożenie ustawionego stanu starszej części portu P5i odczytanej młodszej części portu P7 w jeden bajt
    \end{itemize}
    \item uzyskany kod skaningowy należy przekazać do jednego z rejestrów R, a następnie stan rejestru podać na diody portu P1
    \item kod skaningowy powinien być odczytywany w pętli (nieskończonej)
    \item porty P5 i P7 należy zdefiniować jak port P6 na poprzednich zajęciach:\\ P5 EQU 0F8H \ \ \ P7 EQU 0DBH
\end{itemize}
\subsection{Rozwiązanie}
% \lstinputlisting{zad1.a51}
\section{Zadanie 2}
\subsection{Polecenie}
Zamienić kod skaningowy na NKB:
\begin{itemize}
    \item zamiana zgodnie z nadrukiem na klawiszu:\\ 0 – kod 00H, 1 – kod 01H, ... , A – kod 0AH, B – kod 0BH, itd.
    \item zamianę wykonać poprzez
    \begin{itemize}
        \item operacje logiczno-arytmetyczne lub
        \item tablicę przekodowań w pamięci XRAM:uzyskany kod skaningowy staje się młodszą częścią adresu:80XXH, gdzie XX –uzyskany kod skaningowy
    \end{itemize}
\end{itemize}
\subsection{Rozwiązanie}
% \lstinputlisting{zad2.asm}
\section{Zadanie 3}
\subsection{Polecenie}
Kontrola repetycji naciśniętego klawisza:
\begin{itemize}
    \item rozwiązanie 1: wyłączenie repetycji –kod skaningowy przesyłamy do rejestru dopiero po naciśnięciu i następującym potem zwolnieniu klawisza
    \item rozwiązanie 2: repetycja kontrolowana martwą pętlą:uzyskujemy kod skaningowy natychmiast po naciśnięciu klawisza, następne uzyskanie kodu skaningowego po „odczekaniu w martwej pętli” –pętli uzyskiwanej jak podczas mrugania diodami
\end{itemize}
\section{Zadanie 4}
\subsection{Polecenie}
Związanie obsługi klawiatury z wyświetlaczem LCD: połączyć obsługę klawiatury z obsługą wyświetlacza LCDopracowaną na poprzednich zajęciach
\end{document}
